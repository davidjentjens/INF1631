\documentclass{article}

\usepackage[portuguese]{babel}
\usepackage[utf8]{inputenc}
\usepackage{amsmath}
\usepackage{graphicx}
\usepackage{fancyvrb}
\usepackage{fullpage}
\usepackage{float}
\usepackage{amsmath}
\usepackage{color}
\usepackage{spverbatim}


\definecolor{ogreen}{RGB}{60,128,49}

\restylefloat{table}


\title{Estruturas Discretas - Trabalho 1}

\author{Guilherme Berger \and Gabriel Siqueira \and Felipe Luiz}

\date{\today}

\begin{document}
\maketitle

\textbf{Observação}: o código fonte completo de todos os algoritmos em Python também foi enviado em anexo, para melhor leitura.

\section{Teorema 1}

Dado o teorema 1 e sua prova indutiva, deseja-se um algoritmo que, dados $x$, $y$ e $k$, determine o quociente descrito.

\textbf{Teorema 1} ($k$): $x^k - y^k$ é divisível por $x-y$ para quaisquer $x$ e $y$ inteiros e todos os valores de $k$ inteiros e maiorez que zero.

O quociente supracitado é, então, $\frac{x^k - y^k}{x-y}$ nas condições acima mencionadas.

O algoritmo pode ser derivado diretamente da prova indutiva já fornecida.

Do caso base, sabemos que para $k = 1$, o quociente é igual a 1.

Também diretamente da argumentação fornecida, tem-se que:

\begin{gather}
x^{k+1} - y^{k+1} = q_{k+1} * (x - y) \\
q_{k+1} = (x^k + y * q_k)
\end{gather}

\textbf{Algoritmo genérico}:

{\color{ogreen}
\begin{verbatim}
função quociente(x, y, k)
    se k == 1
        retorna 1
    retorna x^(k-1) + y * quociente(x, y, k - 1)
\end{verbatim}
}

\textbf{Implementação em Python}:

{\color{red}
\begin{verbatim}
def quo(x, y, k):
    if k == 1:
        return 1
    return x**(k-1) + y*quo(x, y, k-1)
\end{verbatim}
}

\pagebreak

\textbf{Testes}: A tabela abaixo ilustra os resultados dos testes:

\begin{table}[H]
\centering
\begin{tabular}{l|l|l|l|l|l}
$x$ & $y$ & $k$ & Execuções & Tempo Total & Tempo/Execução \\\hline
2 & 1 & 1 & 26100000 & 5.015 s & 0.000192 ms \\
2 & 1 & 2 & 6500000 & 5.054 s & 0.000778 ms \\
2 & 1 & 3 & 4000000 & 5.084 s & 0.001271 ms \\
2 & 1 & 3 & 3900000 & 5.014 s & 0.001286 ms \\
2 & 1 & 4 & 2800000 & 5.040 s & 0.001800 ms \\
2 & 1 & 5 & 2200000 & 5.068 s & 0.002303 ms \\
2 & 1 & 7 & 1500000 & 5.099 s & 0.003399 ms \\
2 & 1 & 10 & 1000000 & 5.034 s & 0.005034 ms \\
2 & 1 & 15 & 700000 & 5.624 s & 0.008034 ms \\
2 & 1 & 50 & 200000 & 6.170 s & 0.030851 ms \\
3 & 2 & 1 & 25700000 & 5.013 s & 0.000195 ms \\
4 & 3 & 1 & 22900000 & 5.011 s & 0.000219 ms \\
10 & 9 & 1 & 23500000 & 5.019 s & 0.000214 ms \\
100 & 99 & 1 & 25200000 & 5.019 s & 0.000199 ms \\
3 & 2 & 5 & 2000000 & 5.194 s & 0.002597 ms


\end{tabular}
\end{table}

\textbf{Conclusão}: Observando os resultados dos testes executados e através da propria análise do algoritmo utilizado, podemos constatar que o parâmetro $k$ exerce grande influência no tempo de execução.

\section{Teorema 2}

\textbf{Teorema 2} ($k$): o número de números inteiros cujos dígitos pertencem ao conjunto $\{1,2,...,m\}$ de $k$ dígitos diferentes é dado pelo produto $m * (m-1) * ... * (m-k+1)$.

Como o número não pode possuir dígitos repetidos, podemos concluir que $m\geq k$.

\textbf{Caso base} ($k = 1$): Para formar um número inteiro de 1 dígito, basta escolher um dos $m$ dígitos do conjunto e este será o número. Sendo assim, podemos formar $m$ números.

\textbf{Passo indutivo}: Seja um número $d_1d_2...d_{k-1}$, de $k-1$ dígitos, com $N$ possibilidades distintas de formação. Inserimos um novo dígito $d_k$ ao final deste, a fim de obtermos um número com $k$ dígitos. Podemos verificar que, dos $m$ dígitos disponíveis para o problema, $k-1$ já foram utilizados até então. Logo, existem $m-(k-1) = m-k+1$ possibilidades para $d_k$ e $N*(m-k+1)$ possibilidades para nosso número de $k$ dígitos.
Pela hipótese indutiva, temos que $N = m * (m-1) * ... * (m-k+2)$ e, portanto, o conjunto de números formados por $k$ dígitos diferentes, cada dígito dentre $m$ possibilidades, tem $ m * (m-1) * ... * (m-k+2) * (m-k+1)$ elementos.

\textbf{Observação}: nos algoritmos desenvolvidos, a fim de generalizar um dígito e permitir um valor arbitrário de $m$, cada digito é representado por um inteiro de 1 a $m$. Um número gerado pelo algoritmo, por sua vez, é representado por uma lista de dígitos.

\textbf{Algoritmo Genérico}:

{\color{ogreen}
\begin{verbatim}
função numeros(k, m)

    lista = []

    se k == 1
        para i de 1 até m
            lista.push(i)
        retorna lista

    listaAnterior = numeros(k-1,m)

    para i de 0 até tamanho(listaAnterior)
        para j de 1 até m
            se j não existe em listaAnterior[i]
                novoNumero <- listaAnterior[i]
                novoNumero.push(j)
                lista.push(novoNumero)

    return lista
\end{verbatim}
}

\textbf{Implementação em Python}:

{\color{red}
\begin{verbatim}
def numbers(k, m):

    newList = []

    # Caso base
    if k == 1:
        for i in range(m):
            newList.append([i+1])
        return newList

    previousList = numbers(k-1,m)

    # para cada numero de k - 1 digitos
    for i in range(len(previousList)):
        # para cada digito entre 0 e m-1
        for j in range(m):
            # se j+1 ainda nao foi utilizado no numero corrente,
            # adicionar ao final deste e salvar na lista
            if (j+1) not in previousList[i]:
                newNumber = list(previousList[i])
                newNumber.append(j+1)
                newList.append(newNumber)

    return newList
\end{verbatim}
}

\textbf{Exemplo 1}: com m=3 e k=1, esse foi o resultado gerado pela implementação em Python:

{\color{blue}
\begin{spverbatim}
[[1, 2, 3], [1, 2, 4], [1, 3, 2], [1, 3, 4], [1, 4, 2], [1, 4, 3], [2, 1, 3], [2, 1, 4], [2, 3, 1], [2, 3, 4], [2, 4, 1], [2, 4, 3], [3, 1, 2], [3, 1, 4], [3, 2, 1], [3, 2, 4], [3, 4, 1], [3, 4, 2], [4, 1, 2], [4, 1, 3], [4, 2, 1], [4, 2, 3], [4, 3, 1], [4, 3, 2]]
Total: 24

\end{spverbatim}
}

\textbf{Exemplo 2}: com m=12 e k=2, esse foi o resultado gerado pela implementação em Python:

{\color{blue}
\begin{spverbatim}
[[1, 2], [1, 3], [1, 4], [1, 5], [1, 6], [1, 7], [1, 8], [1, 9], [1, 10], [1, 11], [1, 12], [2, 1], [2, 3], [2, 4], [2, 5], [2, 6], [2, 7], [2, 8], [2, 9], [2, 10], [2, 11], [2, 12], [3, 1], [3, 2], [3, 4], [3, 5], [3, 6], [3, 7], [3, 8], [3, 9], [3, 10], [3, 11], [3, 12], [4, 1], [4, 2], [4, 3], [4, 5], [4, 6], [4, 7], [4, 8], [4, 9], [4, 10], [4, 11], [4, 12], [5, 1], [5, 2], [5, 3], [5, 4], [5, 6], [5, 7], [5, 8], [5, 9], [5, 10], [5, 11], [5, 12], [6, 1], [6, 2], [6, 3], [6, 4], [6, 5], [6, 7], [6, 8], [6, 9], [6, 10], [6, 11], [6, 12], [7, 1], [7, 2], [7, 3], [7, 4], [7, 5], [7, 6], [7, 8], [7, 9], [7, 10], [7, 11], [7, 12], [8, 1], [8, 2], [8, 3], [8, 4], [8, 5], [8, 6], [8, 7], [8, 9], [8, 10], [8, 11], [8, 12], [9, 1], [9, 2], [9, 3], [9, 4], [9, 5], [9, 6], [9, 7], [9, 8], [9, 10], [9, 11], [9, 12], [10, 1], [10, 2], [10, 3], [10, 4], [10, 5], [10, 6], [10, 7], [10, 8], [10, 9], [10, 11], [10, 12], [11, 1], [11, 2], [11, 3], [11, 4], [11, 5], [11, 6], [11, 7], [11, 8], [11, 9], [11, 10], [11, 12], [12, 1], [12, 2], [12, 3], [12, 4], [12, 5], [12, 6], [12, 7], [12, 8], [12, 9], [12, 10], [12, 11]]
Total: 132

\end{spverbatim}
}

\textbf{Testes}: A tabela abaixo ilustra os resultados dos testes:

\begin{table}[H]
\centering
\begin{tabular}{l|l|l|l|l}
$m$ & $k$ & Execuções & Tempo Total & Tempo/Execução \\\hline
10 & 5 & 102 & 5.015 s & 49.161 ms \\
10 & 7 & 4 & 6.635 s & 1658.823 ms \\
10 & 10 & 1 & 33.882 s & 33882.317 ms \\
15 & 5 & 8 & 5.184 s & 647.980 ms \\
15 & 6 & 1 & 7.017 s & 7016.968 ms \\
15 & 7 & 1 & 77.290 s & 77289.940 ms \\
20 & 5 & 2 & 6.860 s & 3429.841 ms \\
20 & 6 & 1 & 55.417 s & 55416.994 ms \\
30 & 5 & 1 & 30.049 s & 30049.451 ms \\
100 & 3 & 4 & 5.712 s & 1428.066 ms \\



\end{tabular}
\end{table}

\textbf{Conclusão}: Podemos verificar que tanto o parâmetro $m$ quanto $k$ exercem grande influência no tempo de execução do programa. Tal fato está diretamente ligado com a implementação utilizada, visto que o aumento de $k$ está relacionado com o aumento das chamadas recursivas, e o aumento de $m$, com o aumento das iterações. Para os seguintes valores \{$m$, $k$\} a computação do algoritmo não foi concluída após mais de 5 minutos de execução: \{10,15\}, \{15,10\}, \{20,7\}, \{30,6\}, \{100,4\} e \{1000,3\}.

\section{Teorema 3}

\textbf{Teorema 3} ($k$): Em um campeonato com $n = 2^k$ equipes, sabe-se construir as $2^k - 1$ rodadas de $2^{k - 1}$ jogos onde cada equipe enfrenta uma equipe diferente em cada rodada.

\textbf{Intuição}: em cada rodada, cada equipe participará de um só jogo, e como cada jogo envolve duas equipes, cada rodada terá $n/2 = 2^{k-1}$ jogos. Como cada equipe só joga com uma outra por rodada, deverão existir $n-1 = 2^k -1$ rodadas para que cada equipe possa jogar com todas as outras.

\textbf{Prova}: feita por indução matemática usando $k$ como parâmetro de indução. As equipes serão numeradas de $e_1$ até $e_{2^k} = e_n$.

\textbf{Caso base} ($k = 1$): temos $n = 2$. Haverá $n-1 = 1$ rodada, com $n/2 = 1$ jogo. Esse jogo é $[e_1, e_2]$.

\textbf{Passo indutivo}: podemos assumir que o teorema é válido para um certo $k$ (hipótese indutiva). Desejamos provar que, a partir disso, o teorema se torna válido para $k+1$. Nesse caso, existem $2^{k+1}$ equipes. 

Divide-se as equipes em dois grupos, $A$ e $B$ de igual tamanho: $\{e_1, ..., e_{2^k}\}$ e $\{e_{2^k+1}, ..., e_{2^{k+1}}\}$. Cada grupo tem $2^k$ equipes. Neste primeiro momento, trataremos os dois grupos como dois torneios independentes e simultâneos. Pela hipótese indutiva, sabemos resolver esses dois problemas (idênticos), e geramos portanto dois torneios, cada um com $2^k - 1$ rodadas de $2^{k - 1}$. Como os torneios são simultâneos, a rodada 1 do torneio $A$ ocorrerá ao mesmo tempo que a rodada 1 do torneio $B$, portanto, ao juntar essas rodadas, teremos $2^k$ jogos por rodada. 

Após esse momento inicial, sabemos que cada equipe do grupo $A$ enfrentou todas as outras equipes de seu grupo. O mesmo vale para o grupo $B$. Resta, portanto, cada equipe do grupo $A$ enfrentar todas as equipes do grupo $B$, e vice-versa.

Para fazer isso, teremos mais $2^k$ rodadas de $2^k$ jogos. Para gerar esses jogos, considere os conjuntos ordenados das equipes do grupo A e do grupo B.

\begin{gather}
a_1 = e_1 \\
a_2 = e_2 \\
... \\
a_{2^k} = e_{2^k} \\
\end{gather}

\begin{gather}
b_1 = e_{2^k + 1} \\
b_2 = e_{2^k + 2} \\
... \\
b_{2^k} = e_{2^{k+1}}
\end{gather}

A primeira rodada deste momento teremos os jogos $[a_1, b_1]$, $[a_2, b_2]$, ..., $[a_{2^k}, b_{2^k}]$

Na segunda rodada, ocorrerá uma rotação nos times de $B$, de modo que o primeiro time de $A$ enfrentará o último time de $B$: $[a_1, b_{2^k}]$, $[a_2, b_1]$, ..., $[a_{2^k}, b_{2^k-1}]$

Nas próximas rodadas, continuará ocorrendo essa rotação, até que na $2^k$-ésima rodada deste momento teremos: $[a_1, b_2]$, $[a_2, b_3]$, ..., $[a_{2^k-1}, b_{2^k}]$, $[a_{2^k}, b_1]$

O total destes dois momentos é $(2^k -1) + (2^k) = 2^{k+1} - 1$ rodadas de $2^k$ jogos, o que está de acordo com o teorema.


\textbf{Algoritmo genérico}:

{\color{ogreen}
\begin{verbatim}
# times é argumento opcional, usado pela recursão
função torneio(k, times)
    se times == nulo
        times = [1, ..., 2^k]

    se k == 1
        jogo <- tupla(times[0], times[1])
        retorna [[jogo]]

    grupo1 <- primeira_metade(times)
    grupo2 <- segunda_metade(times)

    # Momento 1 - divide em dois
    t1 <- torneio(k-1, grupo1)
    t2 <- torneio(k-1, grupo2)
    torneio <- funde_rounds(t1, t2)

    # Momento 2 - ciclo
    para z de 0 ate tamanho(grupo1) - 1
        round <- round_vazio
        para i de 0 ate tamanho(grupo1)
            index1 <- i
            index2 <- (i + z) % tamanho(grupo2)
            jogo <- tupla(grupo1[index1], grupo2[index2])
            round.push(jogo)
        torneio.push(round)

    retorna torneio
\end{verbatim}
}


\textbf{Implementação em Python}:

{\color{red}
\begin{verbatim}
# start é um argumento opcional, usado pela recursão
# O retorno eh uma lista de rounds
# Um round eh uma lista de jogos
# Um jogo eh uma tupla indicando os dois times
def tournament(k, start = 1):
    # Caso base
    if k == 1:
        return [[(start, start+1)]]

    # MOMENTO 1 - divide em dois
    
    # Recursao
    t1 = tournament(k-1, start)
    t2 = tournament(k-1, 2**(k-1)+start)

    # Unir os rounds dos dois torneios
    t = []
    for i in range(len(t1)):
        t.append(t1[i] + t2[i])

    # MOMENTO 2 - ciclo
    
    # Geracao das listas de times
    times1 = range(start, 2**(k-1)+start)
    times2 = range(2**(k-1)+start, 2**(k)+start)
    
    # z vai ser a variavel que faz o ciclo
    for z in range(len(times1)):
        # Estamos em um round
        round = []
        for i in range(len(times1)):
            # Estamos em um par dentro do ciclo
            
            # index1 eh simplesmente i
            # index2 muda de modo a fazer o ciclo no segundo grupo
            index1 = i
            index2 = (i + z) % len(times2)

            game = (times1[index1], times2[index2])
            round.append(game)

        # Adicionamos esse round ao conjunto de rounds, t
        t.append(round)

    return t
\end{verbatim}
}

\pagebreak

\textbf{Exemplo}: com k=3, esse foi o resultado gerado pela implementação em Python (em anexo, o código usado para printar):


{\color{blue}
\begin{verbatim}
Round #01
  Game #01: 1 vs 2
  Game #02: 3 vs 4
  Game #03: 5 vs 6
  Game #04: 7 vs 8
Round #02
  Game #01: 1 vs 3
  Game #02: 2 vs 4
  Game #03: 5 vs 7
  Game #04: 6 vs 8
Round #03
  Game #01: 1 vs 4
  Game #02: 2 vs 3
  Game #03: 5 vs 8
  Game #04: 6 vs 7
Round #04
  Game #01: 1 vs 5
  Game #02: 2 vs 6
  Game #03: 3 vs 7
  Game #04: 4 vs 8
Round #05
  Game #01: 1 vs 6
  Game #02: 2 vs 7
  Game #03: 3 vs 8
  Game #04: 4 vs 5
Round #06
  Game #01: 1 vs 7
  Game #02: 2 vs 8
  Game #03: 3 vs 5
  Game #04: 4 vs 6
Round #07
  Game #01: 1 vs 8
  Game #02: 2 vs 5
  Game #03: 3 vs 6
  Game #04: 4 vs 7
\end{verbatim}
}

\pagebreak

\textbf{Testes}: A tabela abaixo ilustra os resultados dos testes. O maior valor de $k$ para o qual o algoritmo gerou as rodadas foi 13.

\begin{table}[H]
\centering
\begin{tabular}{l|l|l|l}
$k$ & Execuções & Tempo Total & Tempo/Execução \\\hline
1 & 7184461 & 5.000 s & 0.000696 ms \\
2 & 639199 & 5.000 s & 0.007822 ms \\
3 & 180561 & 5.000 s & 0.027692 ms \\
4 & 56125 & 5.000 s & 0.089088 ms \\
5 & 16534 & 5.000 s & 0.302425 ms \\
6 & 4765 & 5.001 s & 1.049445 ms \\
7 & 1138 & 5.004 s & 4.397375 ms \\
8 & 285 & 5.016 s & 17.600289 ms \\
9 & 58 & 5.026 s & 86.647391 ms \\
10 & 15 & 5.331 s & 355.405490 ms \\
11 & 6 & 10.625 s & 1770.771265 ms \\
12 & 6 & 42.569 s & 7094.750086 ms \\
13 & 3 & 85.323 s & 28440.937837 ms 

\end{tabular}
\end{table}

\textbf{Conclusão}: o algoritmo derivado da prova indutiva tem várias limitações. Ele não permite um número arbitrário de equipes, somente potências de 2. A sua execução é demorada e exige que sejam feitas muitas chamadas recursivas, da ordem de $2^k$. Apesar disso, tem resultados corretos e possui uma explicação interessantíssima. Uma característica dele é que é determinístico: não incorpora nenhum fator de aleatoriedade quanto à geração das rodadas e partidas.


\end{document}
