\documentclass[a4paper]{article}

\usepackage[portuguese]{babel}
\usepackage[utf8]{inputenc}
\usepackage{amsmath}
\usepackage{graphicx}
\usepackage{float}
\usepackage{fullpage}
\usepackage{color}
\usepackage[colorinlistoftodos]{todonotes}
\usepackage{spverbatim}
\usepackage{alltt}

% Default fixed font does not support bold face
\DeclareFixedFont{\ttb}{T1}{txtt}{bx}{n}{10} % for bold
\DeclareFixedFont{\ttm}{T1}{txtt}{m}{n}{10}  % for normal

% Custom colors
\usepackage{color}
\definecolor{deepblue}{rgb}{0,0,0.5}
\definecolor{deepred}{rgb}{0.6,0,0}
\definecolor{deepgreen}{rgb}{0,0.5,0}

\usepackage{listings}

% Python style for highlighting
\newcommand\pythonstyle{\lstset{
language=Python,
basicstyle=\ttm,
otherkeywords={self},             % Add keywords here
keywordstyle=\ttb\color{deepblue},
emph={MyClass,__init__},          % Custom highlighting
emphstyle=\ttb\color{deepred},    % Custom highlighting style
stringstyle=\color{deepgreen},
frame=tb,                         % Any extra options here
showstringspaces=false            %
}}


% Python environment
\lstnewenvironment{python}[1][]
{
\pythonstyle
\lstset{#1}
}
{}



\restylefloat{table}
\definecolor{ogreen}{RGB}{60,128,49}

\title{Estruturas Discretas - Trabalho 2}

\author{Felipe Luiz \and Gabriel Siqueira \and Guilherme Berger}

\date{\today}

\begin{document}
\maketitle


%%%%%%%%%%%%%
% TEOREMA 1 %
%%%%%%%%%%%%%
\section*{Teorema 1}

%%%%%%%%%%%%%
\subsection*{Teorema e Prova}

\textbf{Teorema} \:
\textit{Sabe-se encontrar a árvore de peso máximo de $G=(V,E)$ que contém o vértice 1 e possui $K$ vértices}

\medskip
\noindent
\textbf{Prova} \:
Por indução simples em $k$.

Denomina-se $A_k$ a árvore obtida com certo valor de $k$. Denominamos $V_k$ e $E_k$ as listas de vértices e arestas, respectivamente, que compõem a árvore $A_k$.

Para o caso base $k=1$, haverá somente o vértice 1 e nenhuma aresta; o peso total será 0. Esta é a única árvore possível de 1 vértice e que contém $v_1$. Está definida por $V_1 = \{v_1\}$ e $E_1 = \emptyset$.

Pela hipótese indutiva, temos que o teorema é válido para $k$ vértices e desejamos provar, portanto, que é válido também para $k+1$ vértices. Portanto, conhecemos $V_k$ e $E_k$, e deseja-se determinar $V_{k+1}$ e $E_{k+1}$.

Considere o grafo $B_k$ formado pelos vértices pertencentes a $V - V_k$ e por todas as arestas formadas por vértices $(b_1, b_2) \in (V - V_k)$. Considere o conjunto $R$ de arestas do tipo $(a, b)$ em que $a \in A_k$ e $b \in B_k$. Necessariamente, $A_{k+1}$ tem seu conjunto de vértices definido por $V_k \cup \{b\}$ e seu conjunto de arestas definido por $E_k \cup \{(a, b)\}$.

Determinando $a$ e $b$, portanto, determinamos inteiramente $A_{k+1}$.

$a$ e $b$ são os vértices da aresta de maior peso entre as arestas $R$.

Com isso, está determinado $A_{k+1}$.

%%%%%%%%%%%%%
\subsection*{Algoritmo e Código}

Abaixo, mostramos o algoritmo em \textbf{pseudocódigo}:

{\color{ogreen}
\begin{verbatim}
função t1(G, K)

    se K == 1
        retorna uma árvore contendo somente o vértice 1

    A <- t1(G, K-1)
    B <- grafo formado pelos vertices de G nao presentes em A, e por todas as arestas entre esses vertices
    R <- arestas do tipo (a,b) onde ( pertence(a, A) && pertence(b, B))

    new_edge <- elemento de R com maior peso
    a, b <- vertices da aresta new_edge

    adiciona vertice (b) ao grafo (A)
    adiciona aresta (a, b) ao grafo (A)

    retorna A
\end{verbatim}
}

Abaixo, mostramos o algoritmo implementado em \textbf{Python}:

\begin{python}
from pygraph.classes.graph import graph

def teo_1(g, k):
  # Salvaguarda
  if k > len(g.nodes()):
    raise ValueError('FORBIDDEN: K > |V|')
  if k <= 0:
    raise ValueError('FORBIDDEN: K <= 0')

  # Caso base
  if k == 1:
    tree = graph()
    tree.add_node(1)
    return tree

  # Hipotese indutiva
  tree = teo_1(g, k-1)

  # V - V_k
  all_nodes = g.nodes()
  used_nodes = tree.nodes()
  external_nodes = [node for node in all_nodes if node not in used_nodes]

  # Conjunto F de arestas possiveis
  r = []
  for used_node in used_nodes:
    for external_node in external_nodes:
      if g.has_edge((used_node, external_node)):
        r.append((used_node, external_node))

  # Aresta de maior peso em F
  new_edge = max(r, key=lambda e: g.edge_weight(e))

  # V_{k+1} = V_k + {b}
  a, b = new_edge
  tree.add_node(b)

  # E_{k+1} = E_k + {(a, b)}
  tree.add_edge(new_edge)

  return tree
\end{python}

%%%%%%%%%%%%%
\subsection*{Testes e Resultados}


A tabela abaixo ilustra o tempo de execução do algoritmo, em milissegundos, para diferentes instâncias e com diferentes valores do parâmetro $k$.

O tempo foi medido executando o algoritmo por 5 segundos e contando o número de execuções.

A leitura do arquivo e criação da árvore que o representam não foram incluídos no loop.

\begin{table}[H]
\centering
\begin{tabular}{l|l|l|l|l|l|l|l|l|l|l|l|l}
entrada & $k=1$ & $k=2$ & $k=3$ & $k=5$ & $k=10$ & $k=15$ & $k=20$ & $k=30$ & $k=40$ & $k=50$ & $k=|V|$ \\\hline
ulysses16 & 0.003 & 0.031 & 0.069 & 0.184 & 0.554 & 0.979 & -- & -- & -- & -- & 0.989 \\
ulysses22 & 0.004 & 0.037 & 0.113 & 0.261 & 0.890 & 1.74 & 2.12 & -- & -- & -- & 2.16 \\
bays29 & 0.004 & 0.048 & 0.121 & 0.339 & 1.27 & 2.53 & 4.26 & -- & -- & -- & 5.71 \\
eil51 & 0.004 & 0.074 & 0.194 & 0.587 & 2.38 & 5.10 & 8.60 & 16.2 & 26.4 & 33.1 & 33.3 \\
eil51 & 0.004 & 0.074 & 0.194 & 0.587 & 2.38 & 5.10 & 8.60 & 16.2 & 26.4 & 33.1 & 33.3 \\
bier127 & 0.006 & 0.171 & 0.489 & 1.81 & 9.20 & 23.6 & 40.0 & 85.6 & 147.7 & 206.4 & 680.2 \\
lin318 & 0.006 & 0.386 & 1.13 & 3.77 & 17.1 & 51.8 & 116 & 266 & 467 & 722 & 9837 \\

\end{tabular}
\end{table}

No anexo enviado estão figuras que ilustram o passo-a-passo dos vértices e arestas escolhidos em cada etapa do algoritmo quando aplicado na instância ulysses16.


%%%%%%%%%%%%%
% TEOREMA 2 %
%%%%%%%%%%%%%
\section*{Teorema 2 (Bônus)}

%%%%%%%%%%%%%
\subsection*{Teorema e Prova}

\textbf{Teorema} \:
\textit{Sabe-se encontrar a floresta de peso mínimo de de $G=(V,E)$ onde os componentes conexos possuem pelo menos $K$ vértices}

\medskip
\noindent
\textbf{Prova} \:
Por indução simples em $k$. Para o caso base $k = 1$, a floresta $F_1$ conterá todos os vértices de $V$, porém nenhuma aresta. Assim, haverá $|V|$ componentes conexas e a soma dos pesos será mínima.

Pela hipótese indutiva, temos que o teorema é válido para $k$ vértices e desejamos provar, portanto, que é válido também para $k+1$ vértices. Podemos definir componente conexo como qualquer árvore $A=(V',E')$ tal que $V'\subset V$, $E'\subset E$ e $|E'|>0$. Pela hipótese indutiva, um componente conexo de $F_k$ possuirá pelo menos $k$ vértices.
Sendo assim, o único modo de garantir que este componente passe a conter pelo menos $k+1$ vértices é adicionando um novo vértice a este componente.

Então, enquanto houverem componentes conexos em $F_{k+1}$ com número de vértices menores que $k+1$, devemos, do conjunto de arestas de $G$ ainda não utilizadas em $F_{k+1}$ (\textit{i.e.,} $S_k=E-E_{k+1}$), para um componente conexo A de $F_{k+1}$, escolher a aresta de menor peso $s=(v_1,v_2)\in S_k$ tal que $v_1 \in A$ e $v_2 \notin A$. Após a inclusão dessa aresta, o conjunto de componentes conexos deve ser re-avaliado.

Desta maneira, todo componente conexo contará com pelo menos $k+1$ vértices e, assim, obteremos $F_{k+1}$, provando o teorema.


%%%%%%%%%%%%%
\subsection*{Algoritmo e Código}

O algoritmo derivado desta prova indutiva é conhecido como \textit{Algoritmo de Borůvka} e é utilizado para se obter a Árvore Geradora Mínima de grafos ponderados cujos pesos das arestas são distintos.

Abaixo, mostramos o algoritmo em \textbf{pseudocódigo}:

{\color{ogreen}
\begin{verbatim}
função t2(V, E, K)

    se K == 1
        retorna uma floresta contendo todos os vértices, mas nenhuma aresta

    F <- t2(V, E, K-1)

    enquanto houver componente conexa de F com |vertices| < K
        C <- uma componente conexa qualquer de F
        E <- aresta mínima qualquer que não pertence a F com um vértice em F
        adicionar E a F, inclusive seu vértice que não estava em F

    retorna F
\end{verbatim}
}

Abaixo, mostramos o algoritmo implementado em \textbf{Python}:

\begin{python}
from pygraph.classes.graph import graph
from pygraph.algorithms.accessibility import connected_components

def teo_2(g, k):
  # Salvaguarda
  if k > len(g.nodes()):
    raise ValueError('FORBIDDEN: K > |V|')
  if k <= 0:
    raise ValueError('FORBIDDEN: K <= 0')

  # Caso base
  if k == 1:
    forest = graph()
    for node in g.nodes():
      forest.add_node(node)
    return forest

  # Hipotese indutiva
  forest = teo_2(g, k-1)

  # Enquanto ainda houverem componentes conexos
  # que nao satisfazem a condicao
  while True:
    # Atualiza a lista de componentes, pois pode ter
    # mudado durante a adicao
    cc = _transform_cc(connected_components(forest))

    # Seleciona um que tenha comprimento < k
    selected_component = None
    for component in cc:
      if len(component) < k:
        selected_component = component
        break

    # Se nao conseguiu selecionar, significa que todos
    # satisfazem comprimento >= k, e podemos parar o while
    if selected_component == None:
      break

    # Caso haja um selecionado, selecionar a aresta de menor
    # peso que tenha somente um dos vertices em selected_component
    edges = g.edges()
    used_edges = forest.edges()
    unused_edges = [e for e in edges if e not in used_edges]
    neighbor_edges = [e for e in unused_edges if e[0] in selected_component]

    min_edge = min(neighbor_edges, key=lambda e: g.edge_weight(e))
    forest.add_edge(min_edge)

  return forest


def _transform_cc(cc):
  """
  The "connected components" structure returned
  by the function in pygraph is a dict mapping each
  node to an id.
  We'll make a new structure which is a list of lists
  of nodes that are in the same connected component.
  """
  inv_map = {}
  for k, v in cc.iteritems():
    inv_map[v] = inv_map.get(v, [])
    inv_map[v].append(k)
  return inv_map.values()
\end{python}

%%%%%%%%%%%%%
\subsection*{Testes e Resultados}

A tabela abaixo ilustra o tempo de execução do algoritmo, em milissegundos, para diferentes instâncias e com diferentes valores do parâmetro $k$.

O tempo foi medido executando o algoritmo por 5 segundos e contando o número de execuções.

A leitura do arquivo e criação da árvore que o representam não foram incluídos no loop.

\begin{table}[H]
\centering
\begin{tabular}{l|l|l|l|l|l|l|l|l|l|l}
entrada & $k=1$ & $k=2$ & $k=3$ & $k=5$ & $k=10$ & $k=15$ & $k=20$ & $k=30$ & $k=40$ & $k=50$ \\\hline
ulysses16 & 0.01 & 1.41 & 1.99 & 2.31 & 3.35 & 3.64 & -- & -- & -- & -- \\
ulysses22 & 0.01 & 3.09 & 4.66 & 6.36 & 23.57 & 24.95 & 25.02 & -- & -- & -- \\
bays29 & 0.01 & 7.37 & 11.36 & 14.04 & 21.32 & 21.58 & 21.86 & -- & -- & -- \\
eil51 & 0.02 & 59.49 & 98.04 & 148.13 & 163.11 & 170.35 & 173.74 & 175.71 & 175.82 & 183.72 \\
eil76 & 0.03 & 284 & 393 & 539 & 610 & 770 & 818 & 855 & 861 & 865 \\
bier127 & 0.06 & 1861 & 3127 & 3917 & 6735 & 8142 & 10210 & 10352 & 10996 & 13077

\end{tabular}
\end{table}

No anexo enviado estão figuras que ilustram o passo-a-passo dos vértices e arestas escolhidos em cada etapa do algoritmo quando aplicado na instância ulysses16.

%%%%%%%%%%%%%
% TEOREMA 3 %
%%%%%%%%%%%%%
\section*{Teorema 3}

%%%%%%%%%%%%%
\subsection*{Teorema e Prova}

\textbf{Teorema 3}
\textit{(i,j,q): Sabe-se determinar o prêmio máximo que o rei consegue coletar saindo da posição $(i,j)$ e consumindo $q$ unidades.}

Vamos considerar um desarrolamento da matriz em 64 vértices distintos, com $v_1$ correspondente a $(1,1)$, $v_2$ a $(1,2)$, assim por diante. Os conceitos de vizinhança continuam valendo: $v_1$ tem como vizinhos $\{v_2, v_9, v_10\}$.

Considere, também, uma tabela cujas linhas correspondem aos vértices $v$, e as colunas ao custo $q$ restante a ser utilizado. As células da tabela serão preenchidas com o prêmio máximo $P_{max}(v_{ij},q)$, que se consegue a partir de um trajeto que inicie no vértice $v_{ij}$ e que consuma $q$ unidades.

\noindent
\textbf{Prova} \:
Por indução em $q$. Para o caso base $q = 0$, preencheremos a primeira coluna da tabela. Neste caso, não existem unidades para consumir, logo não poderemos sair da origem $(i,j)$. Sendo assim, o prêmio máximo para ir até $(i,j)$ será zero e para qualquer outro vértice será $-\infty$ (que representa a impossibilidade).
Como hipótese indutiva, temos que o teorema é válido para $0 \leq q \leq Q$, e queremos provar que é válido também para $Q+1$. Neste caso, para cada um dos vértices $v$, devemos encontrar o prêmio máximo que pode ser obtido chegando a $v$ consumindo $Q+1$ unidades. Logo, podemos observar que, para que a condição acima seja satisfeita, no instante imediatamente anterior à chegada em $v$, estaríamos em um vértice $v_n$, vizinho de $v$, com $Q+1-q_v$ unidades consumidas, sendo $q_v$ o custo associado ao vértice $v$. Visto que o prêmio $p_v$ associado ao vértice $v$ é constante, devemos escolher $v_n$ de maneira que $P_{max}(v_n,Q+1-q_v)$ seja máximo, garantindo, assim, que $P_{max}(v,Q+1) = p_v + P_{max}(v_n,Q+1-q_v)$ também seja máximo. Vale ressaltar que, caso $Q+1-q_v < 0$, teremos que $P_{max}(v_n,Q+1-q_v) = -\infty$, uma vez que é impossivel chegar a qualquer vértice consumindo um custo total menor que zero.


%%%%%%%%%%%%%
\subsection*{Algoritmo e Código}

Abaixo, mostramos o algoritmo em \textbf{pseudocódigo}:

{\color{ogreen}
\begin{alltt}
função caminhoDePremioMax(v, q)

    se q é zero
        se v é origem
            return caminho({v})
        caso contrario
            return "caminho impossivel"

    se q < 0
        return "caminho impossivel"

    Vn \(\Leftarrow\) conjunto de vizinhos de v

    caminho \(\Leftarrow\) maxPremio\{ caminhoDePremioMax( vn \(\in\) Vn, q-custo(v) )

    caminho.adicionaAoFinal(v)

    return caminho

\end{alltt}
}

Abaixo, mostramos o algoritmo em \textbf{Python}:

\begin{python}
def teo_3(pos, costs, rewards, energy):
  # Salvaguarda
  if pos < 0 or pos >= 64 or energy < 0:
    raise ValueError("Invalid position/energy!!")

  # Dict cujas chaves sao tuplas (posicao, energia) e cujos valores
  # sao tuplas contendo o premio maximo que o rei consegue coletar
  # comecando da posicao dada, com dada energia disponivel, e parando na
  # posicao 0 com 0 energia, e o caminho para tal
  memo = {}

  # Caso base -- q=0
  memo[(0, 0)] = (0, [0])
  for v in range(1, 64):
    memo[(v, 0)] = None

  # Hipotese indutiva e passo indutivo -- preencher a tabela
  # Para cada coluna de energia
  for q in range(1, energy+1):
    # Para cada vertice nessa coluna
    for v in range(64):
      # custo_vizinho e a coluna em que vamos olhar
      custo_vizinho = q - costs[v]
      vizinhos = find_neighbors(v)
      if custo_vizinho < 0:
        memo[(v, q)] = None
      else:
        # Filtra as celulas -- somente se nao for impossivel (not None)
        # e pega a tupla (premio, caminho) delas
        tuplas = [memo[(vizinho, custo_vizinho)] for vizinho in vizinhos if not memo[(vizinho, custo_vizinho)] == None]
        if len(tuplas) > 0:
          # O melhor_vizinho e o que tem maior premio
          melhor_vizinho = max(tuplas, key=lambda x: x[0])
          # O novo_premio e o premio do melhor vizinho somado ao do vertice em questao
          novo_premio = melhor_vizinho[0] + rewards[v]
          # O novo_caminho e o caminho do melhor vizinho acrescido do vertice em questao
          novo_caminho = melhor_vizinho[1][:] + [v]
          memo[(v, q)] = (novo_premio, novo_caminho)
        else:
          memo[(v, q)] = None

  # Com a tabela em maos, vamos encontrar o caminho comecando em 0
  # que obtenha o maior premio possivel, utilizando qualquer quantidade
  # de energia menor que a fornecida
  maior = 0
  for q in range(energy, -1, -1):
    x = memo[(0, q)]
    if x != None and x[0] > maior:
      maior = x[0]
      inst = x + (q,)
  return inst


def find_neighbors(pos):
  x = pos/8
  y = pos%8

  naive = [(x-1, y-1), (x-1, y), (x-1, y+1),
           (x,   y-1),           (x,   y+1),
           (x+1, y-1), (x+1, y), (x+1, y+1)]
  filtered = [n for n in naive if n[0]>=0 and n[1]>=0 and n[0]<8 and n[1]<8]

  return map(lambda pos: pos[0]*8+pos[1], filtered)
\end{python}



%%%%%%%%%%%%%
\subsection*{Testes e Resultados}

Testamos o algoritmo nas instâncias fornecidas. Os resultados encontram-se abaixo, e também em anexo no arquivo walk.out no formato pedido.


\begin{table}[H]
\centering
\begin{tabular}{p{1.5cm}|p{1.5cm}|p{1.5cm}|p{1.5cm}|p{1.5cm}|p{4cm}}
instância & tempo & prêmio obtido & energia utilizada & energia disponível & caminho encontrado \\\hline
1ª & 3.70ms & 16 & 8 & 8 & 0 1 0 1 0 1 0 1 0 \\
2ª & 3.52ms & 16 & 8 & 8 & 0 1 0 1 0 1 0 1 0 \\
3ª & 3.97ms & 16 & 8 & 8 & 0 1 0 1 0 1 0 1 0 \\
4ª & 14.22ms & 284 & 22 & 22 & 0 9 18 27 36 44 52 60 52 60 52 60 52 60 52 60 52 44 36 27 18 9 0 \\
5ª & 9.28ms & 57 & 18 & 18 & 0 8 16 25 16 25 16 25 16 25 16 25 16 25 16 25 16 8 0

\end{tabular}
\end{table}


\end{document}